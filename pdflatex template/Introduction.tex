\chapter{Introduction}
\newline

Virtual Network Function (VNF) placement in edge computing environments is a critical challenge that needs to be addressed to optimize network performance and resource utilization. The efficient placement of VNFs can significantly impact the performance of Service Function Chains (SFCs) and improve the overall network resilience. In this context, a VNF placement strategy that considers the main factors for SFCs and NFVs, such as Energy, CPU, Memory, Bandwidth, Delay, and Distance from the sink nodes in a respective cluster, can help in finding the best place for the VNFs on the edge server of the networks.

In addition, the algorithm used for VNF placement can cluster each sector based on the number of sink nodes, energy situation of the sink node, and bandwidth of the edge cloud. Moreover, the user and network congestion situation of the designated area can also be considered in the VNF placement strategy. Assuming that the coverage of the edge clouds is the same, the proposed strategy aims to find an optimal placement for VNFs to reduce network congestion and improve the network resilience during disasters.

In disaster scenarios, sink nodes are critical components of the network infrastructure that remain safe and continue to provide connectivity to the core network. Therefore, the proposed VNF placement strategy considers the distance from the sink nodes in each cluster to optimize the performance and resilience of the network.

Overall, the proposed VNF placement strategy offers a promising approach to improve the efficiency and resilience of edge computing networks. By considering the key factors that impact the performance of SFCs and NFVs, the strategy aims to optimize the placement of VNFs and reduce the impact of disasters on the network infrastructure.

