\thispagestyle{plain}

\section*{Abstract}
The 5Th generation of the networks (5G) is aim to address the future networks, as this network ability to connect the networks from any kind of the nodes like smart vehicles to cloths or any other objects. 

By enabling high-speed data transfer, low latency, and widespread connection for a variety of devices and  applications, the deployment of 5G networks is not predictable. However, a major issue that must be solved is how to make 5G cellular access networks resilient in the face of disasters. Emerging technologies like O-RAN, MEC, SON, and network slicing integration can be quite useful in this situation for enhancing the resilience of 5G networks in disaster scenarios.

In order to increase the network's resilience in the event of a disaster, we suggest in this paper that VNF placement strategies be used in edge computing environments to maximize the efficient use of network resources. In order to maximize the use of network resources, we also recommend using network or edge cloud and clustering options for VNF placement as the host of the service function chain (SFC). To further increase the disaster resilience of 5G networks, I also assess the possibilities for NFV sharing between the SFCs.

Although they are still in their early stages, O-RAN, MEC, SON, and network slicing integration provides many opportunities for improving the disaster resilience of 5G networks. To improve network resilience and reduce the effect of disasters on the network infrastructure, we, therefore, want to explore the possibilities of these technologies in our suggested VNF placement plan. This study contributes to current attempts to create efficient disaster management solutions in 5G networks, a crucial topic of study in the communications field.







