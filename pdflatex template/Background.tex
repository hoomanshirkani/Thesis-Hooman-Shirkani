\chapter{Background}
\label{chap:background}
Je nachdem wie sehr das Thema in die Tiefe geht, kann es notwendig sein, Hintergrundwissen hier zu beschreiben. Kann auch als Unterkapitel der Einleitung gestaltet werden.

-----------
\newline
% ---NFV (Network Function Virtualization) orchestration in 5G edge computing can be very useful for disaster recovery scenarios. NFV is a technique that involves virtualizing network functions, such as firewalls, load balancers, and routers, which are typically implemented using dedicated hardware.

% In 5G edge computing, NFV can be used to deploy network functions on edge nodes, which are located closer to end-users and devices. This can reduce latency and improve network performance, especially for applications that require real-time processing.

% In the context of disaster recovery, NFV orchestration can help to quickly deploy and configure network functions on edge nodes in response to a disaster. For example, if a natural disaster such as a hurricane or earthquake disrupts communication networks, NFV can be used to quickly deploy and configure network functions such as firewalls, load balancers, and routers on edge nodes in the affected area.

% This can help to restore communication services and provide critical connectivity for emergency responders and affected communities. Additionally, NFV can be used to dynamically allocate network resources based on demand, which can help to optimize network performance and ensure that critical services are prioritized during a disaster.

% Overall, NFV orchestration in 5G edge computing can be a powerful tool for disaster recovery scenarios, enabling faster response times, improved network performance, and better resource allocation.---

% ---To be prepared for network partitioning in 5G disaster recovery with the help of NFV, here are some steps that can be taken:

% Identify critical services and network functions: Identify the critical services and network functions that need to be available during a disaster. This will help to determine which network functions need to be virtualized and deployed on edge nodes to create independent network segments.

% Develop a disaster recovery plan: Develop a disaster recovery plan that outlines the steps to be taken in the event of a disaster, including the process for activating network partitioning using NFV. The plan should include procedures for redirecting network traffic to alternative network segments and restoring network services.

% Select appropriate NFV technologies: Select appropriate NFV technologies and vendors that can support the deployment of network functions on edge nodes to create independent network segments. Consider factors such as scalability, flexibility, and ease of deployment.

% Test and validate the disaster recovery plan: Test and validate the disaster recovery plan to ensure that it works as expected. Conduct regular drills and exercises to ensure that network partitioning can be activated quickly and efficiently in the event of a disaster.

% Ensure redundancy and backup: Ensure that there is redundancy and backup for critical network functions and services. This can include deploying duplicate network functions on different edge nodes and implementing backup power and connectivity options.

% By following these steps, organizations can be better prepared for network partitioning in 5G disaster recovery with the help of NFV. This can help to ensure network availability and resilience during a disaster, and minimize downtime and disruption to critical services.---

% ---In 5G for disaster recovery, the following core function services can be relocated from the core to the edge:

% Traffic routing and load balancing: By relocating traffic routing and load balancing functions to the edge, network traffic can be optimized for better performance and resiliency during a disaster.

% Security functions: Security functions such as firewalls, intrusion detection and prevention systems, and encryption can be relocated to the edge to provide better protection against network threats and attacks.

% Content caching: Content caching functions can be relocated to the edge to enable faster content delivery and reduce network latency, especially during periods of high network traffic.

% Network slicing and service orchestration: Network slicing and service orchestration functions can be relocated to the edge to enable more efficient and flexible network management during a disaster.

% Application-specific functions: Application-specific functions such as video transcoding, image processing, and voice recognition can be relocated to the edge to enable faster processing and reduce latency.

% By relocating these core function services to the edge, organizations can improve network performance, availability, and resilience during a disaster. The edge can provide better proximity to end-users, devices, and critical network infrastructure, enabling faster response times and reducing network latency. Overall, 5G edge computing can be a powerful tool for disaster recovery, enabling organizations to quickly respond to and recover from disruptions to critical network services and infrastructure.---

% ---
% In 5G architecture for disaster recovery, the following functions can be relocated from the 5G core to the edge:

% User plane functions: User plane functions such as packet processing, routing, and forwarding can be relocated to the edge to enable faster and more efficient processing of network traffic. This can help to ensure that critical network services remain available during a disaster.

% Virtualized Radio Access Network (vRAN) functions: By deploying vRAN functions at the edge, organizations can improve the performance and resiliency of the 5G radio access network. This can help to ensure that mobile devices can continue to connect to the network and access critical network services during a disaster.

% Network slicing and service orchestration: By deploying network slicing and service orchestration functions at the edge, organizations can enable more efficient and flexible network management during a disaster. This can help to ensure that critical network services remain available and that network resources are used more efficiently.

% Virtualized Evolved Packet Core (vEPC) functions: By deploying vEPC functions at the edge, organizations can enable faster and more efficient processing of network traffic. This can help to ensure that critical network services remain available during a disaster and that network resources are used more efficiently.

% Edge computing and caching functions: By deploying edge computing and caching functions, organizations can enable faster processing of applications and content at the edge. This can help to reduce network latency and improve the delivery of critical network services during a disaster.
% ---

